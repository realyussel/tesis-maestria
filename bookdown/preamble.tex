\usepackage{booktabs}
% Si está presente: coloque \usepackage{glosarios} y \makeglossaries después de
\usepackage{hyperref}
\usepackage{hyperref}
\usepackage{fancyhdr}
% Paquetes necesarios para los caracteres e idioma en español
\usepackage[utf8]{inputenc}
\usepackage[spanish]{babel}
\selectlanguage{spanish}
\languageshorthands{spanish}
\usepackage{ae}
\usepackage{textcomp}
\usepackage{amsmath}
\usepackage[T1]{fontenc}
% Contiene la definición y estructura de APA
\usepackage{apacite}

% Paquete necesario para el glosario y los acrónimos
% * acronym: para poner los acrónimos definidos en una lista separada,
% * xindy: es una "herramienta de indexación superior" en vez de makeindex,
% * toc: para que el glosario aparezca en su tabla de contenido,
% \usepackage[nomain,acronym,xindy,toc]{glossaries}
\usepackage[acronym]{glossaries}
% Compilar el glosario y los acrónimos
\makeglossaries
% Se incluye el archivo de definición de glosario
% \newglossaryentry{prevalencia}{
name=prevalencia,
description={En epidemiología, proporción de personas que sufren una, enfermedad con respecto al total de la población en estudio.}
}
\newglossaryentry{latex}
{
    name=latex,
    description={LaTeX (short for Lamport TeX) is a document preparation system. The user has to think about only the content to put in the document and the software will take care of the formatting. }
}

% \bibliographystyle{apacite}
% Corrige las referencias al usar apacite, remplaza "y" con "&" y "cols" con "et al".
\renewcommand{\BOthers}[1]{et al.\hbox{}}
\renewcommand{\BBAA}{\&}
\renewcommand{\BBAB}{\&}